\documentclass{article}
\usepackage[letterpaper, total={7.5in, 10in}]{geometry}
\usepackage{setspace}
\usepackage{hyperref}
\usepackage{titlesec}
\usepackage{enumitem}

% Formatting
\titleformat{\section}{
  \vspace{-4pt}\scshape\raggedright\Large
}{}{0em}{}[\vspace{-12pt}\centering\rule{7.5in}{0.4pt}\vspace{-12pt}]

\pagenumbering{gobble}

% Custom Commands
\newcommand{\underlinedLink}[2]{
    {\underline{\large \href{#1}{#2}}}
}

\newcommand{\itemInTable}[1]{
    \begin{itemize}
        \list{#1}
    \end{itemize}
}

\newcommand{\tabitem}[2]{
    \begin{tabular}{lp{#1}}
        \textbullet & #2
    \end{tabular} \vspace{0pt}
}

\begin{document}

% Title section
\begin{spacing}{1.4}
    \begin{center}
        {\bf\Huge Arnav Revankar} \\
        {\large Junior Computer Engineer @ NJIT} \\
        \underlinedLink{https://github.com/BumpyTurtle127}{github.com/avr34} $|$ \underlinedLink{mailto:avr33@njit.edu}{avr33@njit.edu} $|$ \underlinedLink{https://linkedin.com/in/avr33}{linkedin.com/in/avr33}
    \end{center}
\end{spacing}
\vspace{-9pt}

% --------------------------SUMMARY--------------------------
\section{Summary}
    \begin{center}
        \begin{tabular}{p{0.96\textwidth}}
            Highly motivated Computer Engineering student with a strong passion for Embedded Systems, Data Science and Machine Learning. Seeking an internship/coop opportunity to apply my skills and gain experience with real world engineering. Proficient in C, C++, Java, and Python, as well as popular linear algebra and machine learning frameworks like NumPy, PyTorch, and Scikit Learn. Eager to contribute to innovative projects and gain hands-on experience in real world engineering. \\
        \end{tabular}
    \end{center}

\section{Education}
    \vspace{-5pt}\begin{center}
        \begin{tabular}{p{0.99\textwidth}}
            {\large \bf New Jersey Institute of Technology} (Sept 2023 - Present) \hfill Cumulative GPA: 3.65 \\
            BS Computer Engineering, Minor in Data Analytics \hfill Newark, NJ
        \end{tabular}
    \end{center}
    \vspace{-10pt}\begin{itemize}
        \item {\bf Coursework:} Circuits and Systems 1 \& 2, Microprocessors, Computer Architecture, Database System Design, Digital Design, and others.
        \item {\bf Activities:} NJIT Solar Car Club (Software Team)
    \end{itemize}

    \vspace{-5pt}\begin{center}
        \begin{tabular}{p{0.99\textwidth}}
            {\large \bf County College of Morris} (Sept 2022 - May 2023) \hfill Cumulative GPA: 3.83 \\
            Challenger Program for High School Students \hfill Randolph, NJ
        \end{tabular}
    \end{center}
    \vspace{-10pt}\begin{itemize}
        \item Completed over two semesters as a part-time student.
    \end{itemize}

\section{Experience} \vspace{4pt}
    \raggedright\begin{tabular}{lr}
        \parbox[l]{3.65in}{\bf{\large{Internship}}} &
        \parbox[r]{3.65in}{\raggedleft March 2025 - Present} \\
        \parbox[l]{3.65in}{\href{https://www.mathnasium.com/math-centers/chatham}{\underline{Mathnasium}}} &
        \parbox[r]{3.65in}{\raggedleft Chatham, NJ}
    \end{tabular}
    \vspace{-10pt}\begin{itemize} 
        \setlength{\itemsep}{-2pt}
        \item Spearheaded a project developing a full-stack Javascript/Node.js application that generates emails for parents of students at a Math Tutoring Center I work for.
        \item Effectively employs a BNF Grammar, that is used to generate syntactically valid text from a set of customizable templates.
        \item Working on a Node.js backend that'll retrieve student information, further reducing required user input.
        \item Currently being used to send 80-100 emails per day, saving 5 centers 20-30 hours a week.
    \end{itemize}

\section{Skills} \vspace{4pt}
    \begin{itemize}
        \setlength{\itemsep}{-2pt}
        \item {\bf Languages:} C, C++, Java, Javascript, Python, SQL (MySQL), MATLAB, \LaTeX
        \item {\bf Software Tools:} KiCAD, FreeCAD, PlatformIO, Excel, MySQL, Git
        \item {\bf Build Tools/Debuggers:} Apache Maven, Apache Ant, CMake, Make, MSVC \& GCC, GDB, Valgrind
        \item {\bf Libraries:} NumPy, PyTorch, Scikit Learn, Matplotlib, Gnuplot, LWGJL
    \end{itemize}

\section{Projects}
    \raggedright\begin{tabular}{lr}
        \parbox[l]{3.65in}{{\bf{\large{NeuralNet}}} \href{https://github.com/avr34/NeuralNet}{(\underline{Github})}} &
        \parbox[r]{3.65in}{\raggedleft Oct 2024 - Feb 2025} \\
    \end{tabular}
    \begin{itemize} \vspace{-5pt}
        \setlength{\itemsep}{-2pt}
        \item Fully-Connected Neural Network Library for Regression and Classification in C.
        \item A pet project to learn Machine Learning algorithms (minimal external libraries used).
        \item Currently uses ReLU activation, more to be added soon.
        \item Backpropagation using Stochastic Gradient Descent in Matrix form.
    \end{itemize}

    \raggedright\begin{tabular}{lr}
        \parbox[l]{3.65in}{\bf{\large{Collision Avoidance System for Cars}}} &
        \parbox[r]{3.65in}{\raggedleft Sept 2023 - Jan 2024} \\
    \end{tabular}
    \begin{itemize} \vspace{-5pt}
        \setlength{\itemsep}{-2pt}
        \item Accelerometer based system that detects hard braking and alerts surrounding drivers accordingly using hazard lights.
        \item Utilized an Arduino Nano Microcontroller board to handle IO.
        \item Presented in NJIT's First Year Design showcase.
    \end{itemize}
    
    \raggedright\begin{tabular}{lr}
        \parbox[l]{3.65in}{\bf{\large{CRT Oscilloscope Repair}}} &
        \parbox[r]{3.65in}{\raggedleft Nov 2024 - Feb 2025} \\
    \end{tabular}
    \begin{itemize} \vspace{-5pt}
        \setlength{\itemsep}{-2pt}
        \item Bought and rebuilt a BK Precision Model 1461, Single Channel Oscilloscope from Craigslist.
        \item Came with several blown capacitors and one shorted resistor affecting the beam's sweep speed.
    \end{itemize}

\end{document}
